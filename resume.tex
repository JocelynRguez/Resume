%%%%%%%%%%%%%%%%%%%%%%%%%%%%%%%%%%%%%%%%%
% Medium Length Professional CV
% LaTeX Template
% Version 2.0 (8/5/13)
%
% This template has been downloaded from:
% http://www.LaTeXTemplates.com
%
% Original author:
% Trey Hunner (http://www.treyhunner.com/)
%
% Important note:
% This template requires the resume.cls file to be in the same directory as the
% .tex file. The resume.cls file provides the resume style used for structuring the
% document.
%
%%%%%%%%%%%%%%%%%%%%%%%%%%%%%%%%%%%%%%%%%

%----------------------------------------------------------------------------------------
%	PACKAGES AND OTHER DOCUMENT CONFIGURATIONS
%----------------------------------------------------------------------------------------

\documentclass{resume} % Use the custom resume.cls style

\usepackage[left=0.75in,top=0.6in,right=0.75in,bottom=0.6in]{geometry} % Document margins
\usepackage{ebgaramond}

\newcommand{\tab}[1]{\hspace{.2\textwidth}\rlap{#1}}
\newcommand{\itab}[1]{\hspace{0em}\rlap{#1}}
\name{Jocelyn Rodriguez} % Your name
\address{170 E. Sixth St \\ Claremont, CA 91711 \\ (831)~525~1449 \\ jroq2015@pomona.edu}
\address{github.com/JocelynRguez \\ linkedin.com/in/jocelyn-rodriguez}
\begin{document}

%----------------------------------------------------------------------------------------
%	EDUCATION SECTION
%----------------------------------------------------------------------------------------

\begin{rSection}{Education}

{\bf Pomona College} \hfill {\em Expected May 2019}
\\ Bachelor of Arts in Computer Science\\
GPA: 3.4
%Member of Eta Kappa Nu \\
%Member of Upsilon Pi Epsilon \\
\end{rSection}

%----------------------------------------------------------------------------------------
%	WORK EXPERIENCE SECTION
%----------------------------------------------------------------------------------------

\begin{rSection}{Relevent Experience}

\begin{rSubsection}{Western University of Health Sciences}{August 2017 - present}{Social Media Intern}{}
 \item Help engage with faculty, administration, students, and alumni through social media posts and work with Student Affairs on other social media projects.
\end{rSubsection}

\begin{rSubsection}{Information Sciences Institute (USC)}{June 2017 - August 2017}{Summer Undergraduate Research Experience (SURE)}{}
\item Implemented data-sharing and basic UI designs for AuntieTuna, a chrome plug-in that detects possible phishing sites through personalized good lists.
\end{rSubsection}
%------------------------------------------------

\begin{rSubsection}{Pomona College Questbridge Chapter}{September 2016 - May 2017}{Social Media Associate}{}
\item Helped launch events/workshops for 12-50 students such as collaborations with the Career Development Office and Quantitative Skills Research Center and updated social media accounts.
\end{rSubsection}

\end{rSection}

%------------------------------------------------------------------------------------
\begin{rSection}{Campus Engagement and Leadership}

\begin{rSubsection}{Google igniteCS}{March 2017 - present}{Student Leader}
\item IgniteCS provides funding and resources for groups of college and university students to make a difference in their local communities through CS mentorship
\end{rSubsection}

%------------------------------------------------

\begin{rSubsection}{Pomona College Residential Housing Staff }{August 2016 - May 2017}{Sponsor}
\item Mentored a group of 12 first year students with a co-sponsor and planned events to foster an inclusive community

\end{rSubsection}

%------------------------------------------------

\begin{rSubsection}{Quantitative Skills Center}{Februray 2017 - May 2017}{Mentor}
\item Mentored intro-level CS courses to currently enrolled students on a one-on-one basis.
\end{rSubsection}

%------------------------------------------------

\end{rSection}
% ENRICHMENT
%
\begin{rSection}{Enrichment Programs}
%
  \begin{rSubsection}{Codesmith Summer Academy of Code}{August 2017}{Student}
  \item A week long residency program where students are immersed in code through a combination of pair-programming, project-based assessments, and mentorship by industry leaders.
  \end{rSubsection}
%
%
\end{rSection}

% CONFERENCES
%
\begin{rSection}{Conferences}
%
  \begin{rSubsection}{Tapia Conference}{September 2016}{Presenter}
  \item Along with two other students, I presented about Pomona College's Learning Community for Underrepresented Students in Computer Science.
  \end{rSubsection}
%
%
\end{rSection}

%----------------------------------------------------------------------------------------
%	TECHNICAL STRENGTHS SECTION
%----------------------------------------------------------------------------------------
\begin{rSection}{Skills}

\begin{tabular}{ @{} >{\bfseries}l @{\hspace{6ex}} l }
Computer Languages &  Java, HTML/CSS, JavaScript, LaTeX, Standard ML, Assembly, C \\
Relevent Courses & Data Structures Adv Programming, Fundamentals of Computer Science, \\ & Computer Systems, Discrete Mathematics, Linear Algebra
% Software \& Tools & HTML, LaTeX, Excel, Gerris, Mathematica, ASPEN Plus, Tecplot \\
\end{tabular}

\end{rSection}



\end{document}
